\documentclass{beamer}

\mode<presentation>
{
  \usetheme{default}      % or try Darmstadt, Madrid, Warsaw, ...
  \usecolortheme{default} % or try albatross, beaver, crane, ...
  \usefonttheme{default}  % or try serif, structurebold, ...
  \setbeamertemplate{navigation symbols}{}
  \setbeamertemplate{caption}[numbered]
} 

\usepackage[brazilian]{babel}
\usepackage[utf8]{inputenc}
\usepackage[T1]{fontenc}
\usepackage{listings}
\usepackage{graphicx}

\title[EP3]{EP3 - MAC0422}
\author {Diego Alvarez, Thiago I. S. Pereira}

\date{\today}

\begin{document}
\begin{lstlisting}[
    basicstyle=\tiny
]
\end{lstlisting}

\begin{frame}
  \titlepage
\end{frame}


\section{Introdução}

\begin{frame}{Introdução}

\begin{block}{}
\par Esta apresentação tem como objetivo exemplificar a implementação do EP3 de MAC0422 - Sistemas Operacionais. \par Iremos apresentar um simulador de sistema de arquivos baseado em FAT, usando bitmaps para gerencia do espaco livre.
\end{block}

\end{frame}

\section{O simulador}

\subsection{O simulador}

\begin{frame}{O Simulador}
\par O simulador foi totalmente implementado e testado ultilizando python3 em ambiente GNU/Linux
\vskip 1cm
\begin{itemize}
\setlength\itemsep{1em}
\item python3 - versão 3.4.2
\item Debian GNU/Linux 8.2 Jessie
\end{itemize}
\end{frame}

\subsection{O Simulador}
\begin{frame}{O Simulador}
O simulador e o shell foram implementados separadamente, o shell chamando o simulador como se fosse um programa separado, cada um tendo seus proprios módulos.
\end{frame}

\subsection{O Simulador}
\begin{frame}{O Simulador}
\par O simulador foi dividos nos seguintes modulos:
\vskip 1cm
\begin{itemize}
\setlength\itemsep{1em}
\item {\bf interface} - Implementa os comandos pedidos no enunciado, se utilizando das outras estruturas para controlar. É o módulo de mais alto nível.
\item {\bf filesystem } - Implementa a gerencia das estruturas principais (FAT e bitmap) e intermedia o acesso ao arquivo físico. É o modulo de mais baixo nível.
\item {\bf directory } - Estrutura que corresponde a um diretorio no sistema de arquivos e possui facilidade para manipular suas entradas
\item {\bf entry } - Estrutura de uma entrada de em diretorio. Mantem os metadados de arquivos e diretorios, sempre estão contidos em um diretorio.
\end{itemize}
\end{frame}

\subsection{O Simulador}
\begin{frame}{O Simulador}
\par O shell foi dividido nos seguintes modulos:
\vskip 1cm
\begin{itemize}
\setlength\itemsep{1em}
\item {\bf prompt} - Verifica a entrada do usuário, se todos os comandos e valores correspondem com o enunciado do EP.
\item {\bf main} - Inicia um shell e com a ajuda do modulo prompt verifica as entradas e executa o simulador com os parametros inseridos
\end{itemize}
\end{frame}

\subsection{Simulação de Sistema de Arquivos}
\begin{frame}{Formato binário}
\begin{itemize}
\setlength\itemsep{1em}
\item \par O sistema de arquivos é composto por 24.986 setores, a FAT ocupa 49.972 bytes e o bitmap 3.124 bytes

\item \par O arquivo binário guarda primeiro o bitmap, seguido pela FAT, seguido pelos setores para armazenamento

\item \par O diretório raiz sempre está nos 3 primeiros setores
\end{itemize}
\end{frame}

\subsection{Simulação de Sistema de Arquivos}
\begin{frame}{Formato binário}
\begin{itemize}
\setlength\itemsep{1em}
\item Os diretorios são formados por uma lista que ocupa 3 blocos (12KB) e comportam um numero fixo de 240 entradas
\item Cada entrada tem um tamanho de 50 bytes e é composta de 31B para o nome, 1B para o tipo, 4B para o tamanho, 3x4B para as datas e 2B de ponteiro para os dados
\item Assim cada arqivo/diretório deve ter no máximo 31 bytes (em utf-8) de nome
\end{itemize}
\end{frame}

\subsection{Testes}
\begin{frame}{Metodologia}
\begin{itemize}
\setlength\itemsep{1em}
\item Foi feito um script para rodar todos os testes automaticamente
\item Foram rodados os 8 testes pedidos para cada um dos estados (vazio, 10MB cheio e 50MB cheio)
\item Cada teste foi repetido 30 vezes e seus resultados estão a seguir
\end{itemize}
\end{frame}


\subsection{Testes}
\begin{frame}{Testes}
\begin{table}[]
\centering
\label{my-label}
\resizebox{\textwidth}{!}{%
\begin{tabular}{lccc}
\multicolumn{4}{c}{Tempo em milisegundos}\\
\hline
\multicolumn{1}{l}{Teste} & \multicolumn{1}{l}{Vazio} & \multicolumn{1}{l}{10MB cheio} & \multicolumn{1}{l}{50MB cheio} \\
\hline
cp 1mb & 7.6 & 13.3& 36.3\\
cp 10mb & 44.2& 98.8& 335.7\\
cp 30mb & 292.6& 408.1& 1160.3\\
rm 1mb & 6.8& 6.5& 6.4\\
rm 10mb & 6.8& 6.8& 6.9\\
rm 30mb & 7.8& 7.7& 8.1\\
rmdir vazio & 35.7& 34.1& 34.4\\
rmdir cheio & 40.4& 40.2& 41.1
\end{tabular}
}
\end{table}
\end{frame}


\subsection{Testes}
\begin{frame}{Análise}
\begin{itemize}
\setlength\itemsep{1em}
\item Mesmo as operacoes simples demoram um pouco devido ao overhead de desmontar o sistema de arquivos
\item A operacao mais lenta é de copiar arquivos, o que é esperado pois há mais dados a serem movidos
\item O estado do sistema de arquivos aparentemente só influencia a velocidade de cópia, o que faz sentido pois essa é a unica operacao que precisa alocar espaco
\item Remover uma arvore de diretorios cheia é apenas um pouco mais lento que remover uma vazia
\end{itemize}
\end{frame}


\end{document}

